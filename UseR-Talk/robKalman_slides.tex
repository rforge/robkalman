% ------------------------------------------------------------------------
% Vortrag: Wien 18-06-06
% ------------------------------------------------------------------------
%
%%%%%%%%%%%%%%%%%%%%%%%%%%%%%%%%%%%%%%%%%%%%%%%%%%%%%%%%%%%%%%%%%%%%%%%%%%%
%
%
%
%%%%%%%%%%%%%%%%%%%%%%%%%%%%%%%%%%%%%%%%%%%%%%%%%%%%%%%%%%%%%%%%%%%%%%%%%%%
% fuer Handout ab %%<H  bis  %%>H einkommentieren
%      (und %%<D, %%>D rauskommentieren)
% fuer dynam.Slides ab %%<D  bis  %%>D einkommentieren
%      (und %%<H, %%>H rauskommentieren)
%%%%%%%%%%%%%%%%%%%%%%%%%%%%%%%%%%%%%%%%%%%%%%%%%%%%%%%%%%%%%%%%%%%%%%%%%%%
%
%%<H
%\documentclass[handout]{beamer}
%%>H
%%%%%%%%%%%%%%%%%%%%%%%%%%%%%%%%%%%%%%%%%%%%%%%%%%%%%%%%%%%%%%%%%%%%%%%%%%%%
%%<D
\documentclass[10pt]{beamer}
%
\mode<presentation>
{
  \usetheme{Warsaw}
%  % oder ...%%%
%
  \setbeamercovered{transparent=20, again covered = \opaqueness<1->{60}}
%  % oder auch nicht
  \usecolortheme{UBT}
}
%%>D
%%%%%%%%%%%%%%%%%%%%%%%%%%%%%%%%%%%%%%%%%%%%%%%%%%%%%%%%%%%%%%%%%%%%%%%%%%%%
%\mode{handout}
%{
%  \usetheme{Warsaw}
%  % oder ...%
%
%  \setbeamercovered{transparent=20, again covered = \opaqueness<1->{60}}
%  % oder auch nicht
%  \usecolortheme{UBT}
%}
%
\usepackage[english]{babel}
\usepackage[latin1]{inputenc}
\RequirePackage{defrbeam}  %%  im Verzeichnis UBTstyle  --- f�r UBT Abk�rzungen
\RequirePackage{listings}  %%  im Verzeichnis listings  --- f�r Programm-St�ckchen
%
%FARBdefs
\definecolor{notrot}{rgb}{0,0,0}%{HTML}{802040}
\definecolor{normal}{rgb}{0,0,0} %{0.9807843,0.8455490,0.3125490}
\definecolor{Rcolor}{rgb}{0.065,0.065,0.485}
%
%%%%%%%%%%%%%%%%%%%%%%%%%%%%%  Spezifikation von <- im R-Code modus
\lstset{language=R,basicstyle=\small\color{Rcolor},literate={<-}{{$\leftarrow$}}2,morekeywords=[2]{Norm,Pois,lambda,p,d,r,distroptions}}
\newcommand{\ttR}[1]{{\color{Rcolor}\tt #1\color{normal}}}
%%%%%%%%%%%%%%%%%%%%%%%%%%%%%%%%%%%%%%%%%%%%%%%%%%%%%%%%%%%%%%%%%%%%%%%%%%
%
\title%[] % (optional, nur bei langen Titeln n�tig)
{{\tt robKalman} --- a package on Robust Kalman Filtering}

\author{Peter Ruckdeschel\inst{1} \and
Bernhard Spangl\inst{2}%
}
\institute
{\inst{1}{\includegraphics[width=1.4cm]{UBTlogo}\hspace{0.3cm}Fakult\"at f\"ur Mathematik und Physik\smallskip\\
\mref{Peter.Ruckdeschel@uni-bayreuth.de}\\
\href{http://www.uni-bayreuth.de/departments/math/org/mathe7/RUCKDESCHEL}{\tiny \tt www.uni-bayreuth.de/departments/math/org/mathe7/RUCKDESCHEL}}
\and \inst{2}{\includegraphics[bb=262 426 291 472,width=0.35cm]{RLI_Logo_klein_CMYK}\hspace{0.3cm}
Universit\"at f\"ur Bodenkultur, Wien\smallskip\\
%Gregor-Mendel-Stra\ss{}e 33\smallskip\\
%A-1180 Vienna\smallskip\\
\mref{Bernhard.Spangl@boku.ac.at}\\
\href{http://www.rali.boku.ac.at/statedv.html}{\tiny \tt www.rali.boku.ac.at/statedv.html}}
}
% - Der \inst{?} Befehl sollte nur verwendet werden, wenn die Autoren
%   unterschiedlichen Instituten angeh�ren.
% - Keep it simple, niemand interessiert sich f�r die genau Adresse.
%
\date{\includegraphics[bb=0 0 236 113,width=1.65cm]{useR-middle}\hspace{0.3cm}%UseR\\%
16.06.2006}
% - Volle oder abgek�rzter Name sind m�glich.
% - Dieser Eintrag ist nicht f�r das Publikum gedacht (das wei�
%   n�mlich, bei welcher Konferenz es ist), sondern f�r Leute, die die
%   Folien sp�ter lesen.


% Falls eine Logodatei namens "university-logo-filename.xxx" vorhanden
% ist, wobei xxx ein von latex bzw. pdflatex lesbares Graphikformat
% ist, so kann man wie folgt ein Logo einf�gen:

%\pgfdeclareimage[height=0.4cm,width=1.8cm]{logo}{ubtlogo}
%\logo{\pgfuseimage{logo}}



% Folgendes sollte gel�scht werden, wenn man nicht am Anfang jedes
% Unterabschnitts die Gliederung nochmal sehen m�chte.
%\AtBeginSubsection[]
%{
%  \begin{frame}<beamer>
%    \frametitle{Outline}
%    \tableofcontents[currentsection,currentsubsection]
%  \end{frame}
%}


% Falls Aufz�hlungen immer schrittweise gezeigt werden sollen, kann
% folgendes Kommando benutzt werden:

%\beamerdefaultoverlayspecification{<+->}



\begin{document}

% ------------------------------------------------------------------------
% ------------------------------------------------------------------------
\begin{frame}%Frame 1
  \titlepage
\end{frame}
%%%%%%%%%%%%%%%%%%%%%%%%%%%%%%%%%%%%%
%
%  aus den Guidelines von Paket beamer
%
%\begin{frame}
%  \frametitle{Outline}
%  \tableofcontents[pausesections]
%  % Die Option [pausesections] k�nnte n�tzlich sein.
%\end{frame}
%
%
% Einen Vortrag zu strukturieren ist nicht immer einfach. Die
% nachfolgende Struktur kann unangemessen sein. Hier ein paar Regeln,
% die f�r diese L�sungsvorlage gelten:
%
% - Es sollte genau zwei oder drei Abschnitte geben (neben der
%   Zusammenfassung).
% - *H�chstens* drei Unterabschnitte pro Abschnitt.
% - Pro Rahmen sollte man zwischen 30s und 2min reden. Es sollte also
%   15 bis 30 Rahmen geben.
%
% - Konferenzteilnehmer wissen oft wenig von der Materie des
%   Vortrags. Deshalb: vereinfachen!
% - In 20 Minuten ist es schon schwer genug, die Hauptbotschaft zu
%   vermitteln. Deshalb sollten Details ausgelassen werden, selbst
%   wenn dies zu Ungenauigkeiten oder Halbwahrheiten f�hrt.
% - Falls man Details wegl�sst, die eigentlich wichtig f�r einen
%   Beweis/Implementation sind, so sagt man dies einmal n�chtern. Alle
%   werden damit gl�cklich sein.
%%%%%%%%%%%%%%%%%%%%%%%%%%%%%%%%%%%%%%%%%%%%%%%%%%%%%%%%%%%%%%%%%%%%%%%%%%%%%%
% ------------------------------------------------------------------------
\section{Robust Kalman Filtering}
% ------------------------------------------------------------------------
\subsection{Classical setup}
% ------------------------------------------------------------------------
\begin{frame}%Frame 2
  \frametitle{Classical setup: Linear state space models (SSMs)}
\begin{itemize}
  \item<1-> State equation:
            $$\framebox{$X_t=F_t X_{t-1} + v_t$}$$
  \item<2-> Observation equation:
            $$\framebox{$Y_t=Z_t X_{t} + \varepsilon_t$}$$
  \item<3-> Ideal model assumption:
  $$X_0\sim {\cal N}_p(a_0,\Sigma_0),\quad v_t\sim {\cal N}_p(0,Q_t),\quad \varepsilon_t\sim {\cal N}_q(0,V_t),$$
   all independent
  \item<4> (preliminary ?) simplification: Hyper parameters $F_t,Z_t,V_t,Q_t$ constant in $t$
\end{itemize}
\end{frame}
% ------------------------------------------------------------------------
\begin{frame}[fragile]%Frame 3
  \frametitle{Problem and classical solution}
\begin{itemize}
 \item<1>Problem: Reconststruction of $X_t$ by means of $Y_s, s\leq t$
 \item<2>Criterium: MSE
 \item<2>$\leadsto\quad$ general solution: $\Ew X_t|(Y_s)_{s\leq t}$
 \item<3>Computational difficulties:\\
  $\;\Longrightarrow\;$ restriction to {\bf linear\/} procedures \newline \hphantom{$\;\Longrightarrow\;$}/ or: Gaussian assumptions
  \item<3> $\leadsto\quad$ classical {\bf Kalman Filter}
\end{itemize}
\end{frame}
% ------------------------------------------------------------------------
\begin{frame}[fragile]%Frame 4
  \frametitle{Kalman filter}
 \begin{enumerate}
   \setcounter{enumi}{-1}
   \item<1-> Initialization ($t=0$): $$X_{0|0}=a_0, \quad \Sigma_{0|0}=\Sigma_0$$
   \item<2-> Prediction ($t\ge 1$): $$X_{t|t-1}=F X_{t-1|t-1}, \quad \Cov(X_{t|t-1})=\Sigma_{t|t-1}=F\Sigma_{t-1|t-1}F'+Q$$
   \item<3> Correction ($t\ge 1$):
   \uncover<3->{
   \begin{eqnarray*}
    X_{t|t}&=&X_{t|t-1}+ K_t (Y_t-Z X_{t|t-1})\\
   K_{t}&=&\Sigma_{t|t-1}Z'(Z\Sigma_{t|t-1}Z'+V)^-, \qquad \mbox{(Kalman gain)}\\
   \Cov(X_{t|t})&=&\Sigma_{t|t}=\Sigma_{t|t-1}-K_t Z\Sigma_{t|t-1}
   \end{eqnarray*}}
 \end{enumerate}
\end{frame}
% ------------------------------------------------------------------------
\subsection{Robustification}
% ------------------------------------------------------------------------
\begin{frame}%Frame 5
  \frametitle{Types of outliers and robustification}
\begin{itemize}
  \item<1> IOs (system intrinsic): state equation is distorted\\ --- not considered here
  \item<2-> AO/SOs (exogeneous): observations are distorted:
\begin{itemize}
  \item<2-> either error $\varepsilon_t$ is affected (AO)
  \item<2-> or observations  $Y_t$ are modified (SO)
 \end{itemize}%
 \item<3-> a robustifications as to AO/SOs is to
 \begin{itemize}
    \item<3-> retain recursivity (three-step approach)
    \item<3-> modify correction step $\leadsto$ bound influence of $Y_t$
    \item<3-> retain init./pred.step but with modified filter past $X_{t-1|t-1}$
 \end{itemize}%
 \end{itemize}%
\end{frame}
% ------------------------------------------------------------------------
\subsection{Approaches}
% ------------------------------------------------------------------------
\begin{frame}%Frame 6
  \frametitle{Considered approaches}
Approximate conditional mean (ACM): \cite{Mar:79}
\begin{itemize}
  \item<1-> $\dim Y_t=1$
  \item<2-> particular model: $Y_t \sim {\rm AR}(p)$\\
   \begin{itemize}
      \item<2-> $\leadsto$ $X_t=(Y_t,\ldots,Y_{t-p+1})$,
      \item<2-> hyper parameters $Z=(1,0,\ldots,0)$, $V^{\rm \scriptsize id}=0$, $F$, $Q$ unknown
   \end{itemize}%
  \item<3-> estimation of $F$, $Q$ by means of {\rm GM}-Estimators
  \item<4-> modified Corr.step:   for suitable location influence curve $\psi$
  \uncover<4->{
\begin{eqnarray*}
    X_{t|t}&=&X_{t|t-1}+\Sigma_{t|t-1}Z'\psi(Y_t-Z X_{t|t-1})\\
   \Sigma_{t|t}&=&\Sigma_{t|t-1}-\Sigma_{t|t-1} Z'\psi'(Y_t-Z X_{t|t-1}) Z \Sigma_{t|t-1}
   \end{eqnarray*}}
 \end{itemize}%
\end{frame}
% ------------------------------------------------------------------------
\begin{frame}%Frame 7
  \frametitle{Considered approaches II}
 rLS filter: \cite{PR01}
  \begin{itemize}
  \item<1->$\dim X_t, \dim Y_t $ arbitrary, finite
  \item<1-> assumes hyper parameters $a_0$, $Z$, $V^{\rm \scriptsize id}$, $F$, $Q$ known
  \item<2-> modified Corr.step:
  \uncover<2->{
  \begin{eqnarray*}
    X_{t|t}&=&X_{t|t-1}+H_b\big(K_t(Y_t-Z X_{t|t-1})\big)\\
   H_b(X)&=&X\min\{1,b/|X|\}\qquad\mbox{for $|\,\cdot\,|$ Euclidean norm}
   \end{eqnarray*}}
  \item<3-> optimality for SO's in some sense
   \end{itemize}%
\end{frame}
% ------------------------------------------------------------------------
\section{Implementation proposal}
% ------------------------------------------------------------------------
\subsection{Concept / Strategy}
% ------------------------------------------------------------------------
\begin{frame}%Frame 8
  \frametitle{Concept and strategy}
Goal: package {\tt  robKalman}\\
Contents
  \begin{itemize}
     \item<2-> Kalman filter: filter, Kalman gain, covariances
     \item<2-> ACM-filter: filter, GM-estimator
     \item<2-> rLS-filter: filter, calibration of clipping height
     \item<3-> further recursive filters?\\
      $\leadsto$ general interface {\tt recursiveFilter}\\
      with arguments:
        \begin{itemize}
           \item state space model (hyper parameters)
           \item functions for the init./pred./corr.step
        \end{itemize}%
  \end{itemize}%
\end{frame}
% ------------------------------------------------------------------------
\begin{frame}%Frame 9
  \frametitle{Concept and strategy II}
\begin{itemize}
  \item<1-> Programming language
  \begin{itemize}
    \item completely in {\tt S}
    \item perhaps some code in {\tt C} (much) later
  \end{itemize}%
  \item<2-> Use existing infrastructure
  \begin{itemize}
     \item  from where to ``borrow'':
      \begin{itemize}
      \item univariate setting: {\tt KalmanLike} (package {\tt stats});\\
      time series classes: {\tt ts}, {\tt its}, {\tt irts}, {\tt zoo}, {\tt zoo.reg}, {\tt tframe}
      \item multivariate setting: {\tt dse} bundle by Paul Gilbert; perhaps {\tt zoo}?
     \end{itemize}
     \item use for: graphics, diagnostics, management of date/time
   \end{itemize}
   \item<3-> Split user interface and ``Kalman code''
   \begin{itemize}
     \item internal functions: no {\tt S4}-objects
     \item user interface: {\tt S4}-objects
   \end{itemize}%
\end{itemize}%
\end{frame}
% ------------------------------------------------------------------------
\begin{frame}%Frame 10
  \frametitle{Concept and strategy III}
\begin{itemize}
     \item<1-> Use of {\tt S4}
   \begin{itemize}
     \item<1-> Hierarchic Classes:
   \begin{itemize}
          \item state space models (SSMs) (Hyper-Parameter, distributional assumptions, outlier types)
          \item filter results (specific subclass of (multivariate) time series)
          \item control structures for filters (tuning parameters)
   \end{itemize}%
     \item<2-> Methods:
          \begin{itemize}
             \item filters (for different types of SSMs)
             \item accessor/replacement functions
             \item {\tt simulate} for SSMs
             \item filter diagnostics: {\tt getClippings}, {\tt conf.intervals} ?
             \item tests?
          \end{itemize}%
     \item<2-> constructors/generating funtions
   \end{itemize}%
\end{itemize}%
\end{frame}
% ------------------------------------------------------------------------
\subsection{Implementation so far}
% ------------------------------------------------------------------------
\begin{frame}%Frame 11
  \frametitle{Implementation so far: interfaces}
\begin{itemize}
  \item<1-> preliminary, ``{\tt S4}-free'' interfaces
  \begin{itemize}
     \item Kalman filter (in our context) {\tt KalmanFilter}
     \item rLS (P.R.): {\tt rLSFilter} \\
           --- with routines for
           calibration at given
          \begin{itemize}
                \item efficency in ideal model
                \item contamination radius
          \end{itemize}%
     \item<2-> ACM (B.S.) {\tt ACMfilt}, {\tt ACMfilter}
     \begin{itemize}
        \item with function {\tt arGM} for AR-parameters by GM-estimates
        \item various $\psi$-functions are available:\\
              Hampel (ACM-filter), Huber,  Tukey (both GM-estimators)\\ ---see {\tt ?.psi}
     \end{itemize}%
     \item<3-> all: wrappers to {\tt recursiveFilter}
  \end{itemize}%
\end{itemize}%
\end{frame}
% ------------------------------------------------------------------------
\begin{frame}%Frame 12
  \frametitle{Implementation so far: package {\tt robKalman}}
\begin{itemize}
  \item<1-> package {\tt robKalman}
  \begin{itemize}
      \item routines gathered in package {\tt robKalman},  version {\tt 0.1}
      \item documentation
      \item demos
  \end{itemize}%
\item<2-> required packages --- all available from \href{http://cran.r-project.org}{\tt CRAN}:
{\tt methods}, {\tt graphics}, {\tt startupmsg}, {\tt dse1}, {\tt dse2}, {\tt MASS}, {\tt limma}, {\tt robustbase}
\item<3-> availability: web-page setup under\bigskip\\
       \fbox{\parbox{9cm}{\href{http://www.uni-bayreuth.de/departments/math/org/mathe7/robKalman/}%
       {{\tt http://www.uni-bayreuth.de/departments/}\\%
       {\hphantom{\tt http:/}{\tt /math/org/mathe7/robKalman/}}}}}
\end{itemize}%
\end{frame}
% ------------------------------------------------------------------------
\subsection{Next steps}
% ------------------------------------------------------------------------
\begin{frame}%Frame 13
  \frametitle{Next steps}
\begin{itemize}
  \item<1-> OOP
  \begin{itemize}
      \item definition of {\tt S4} classes \\ $\leadsto$ close contact to
      \begin{itemize}
            \item {\tt RCore},
             \item Paul Gilbert,
            \item  possibly Gabor Grothendiek and Achim Zeileis ({\tt zoo})
      \end{itemize}
      \item casting/conversion functions for various time series classes
  \end{itemize}%
  \item<2-> User interface {\tt robfilter} (?)
  \begin{itemize}
   \item goal: four arguments: data, SSM, control-structure, filter type
   \item should cope with various definitions of SSMs, data in various time series classes,
   \item possibly simpler interfaces for ACM ({\tt Splus}-compatibility) $\leadsto$ {\tt ACMfilt}-like
  \end{itemize}%
  \item<3-> Release schedule
  \begin{itemize}
      \item wait for results of discussion as to class definition
      \item guess: end of 2006
  \end{itemize}%
 \end{itemize}%
\end{frame}
% ------------------------------------------------------------------------
\section{Demonstration}
% ------------------------------------------------------------------------
\subsection[ACMfilt]{{\tt ACMfilt}}
% ------------------------------------------------------------------------
\begin{frame}[fragile]%Frame 14
  \frametitle{Demonstration: {\tt ACMfilt}}
\begin{lstlisting}
##  generation of data from AO model:
set.seed(361)
Eps <- as.ts(rnorm(100))
ar2 <- arima.sim(list(ar = c(1, -0.9)),
       100, innov = Eps)
Binom <- rbinom(100, 1, 0.1)
Noise <- rnorm(100,sd = 10)
y <- ar2 + as.ts(Binom*Noise)

## determination of GM-estimates
y.arGM <- arGM(y, 3)
## ACM-filter
y.ACMfilt <- ACMfilt(y, y.arGM)

plot(y)
lines(y.ACMfilt$filt, col=2)
lines(ar2,col="green")
\end{lstlisting}
\end{frame}
% ------------------------------------------------------------------------
\begin{frame}%Frame 15
\parbox{13cm}{\hspace{-1cm}\parbox{9cm}{\includegraphics[width=9cm]{ACM}}
\hspace{-1cm}\parbox{4.5cm}{%
\begin{tabular}{lp{3cm}}
green:& ideal time series,\\
black:& AO contam. time series,\\
red:& result ACM
\end{tabular}}%
}
\end{frame}
% ------------------------------------------------------------------------
\subsection[rLSFilter]{{\tt rLSFilter}}
% ------------------------------------------------------------------------
\begin{frame}[fragile]%Frame 16
  \frametitle{Demonstration: {\tt rLSFilter}}

\begin{lstlisting}
## specification of SSM: (p=2, q=1)
a0   <- c(1, 0); S0  <- matrix(0, 2, 2)
F   <- matrix(c(.7, 0.5, 0.2, 0), 2, 2)
Q   <- matrix(c(2, 0.5, 0.5, 1), 2, 2)
Z   <- matrix(c(1, -0.5), 1, 2)
Vi  <- 1;
## time horizon:
TT<-50
## AO-contamination
mc  <- -20; Vc  <- 0.1; ract <- 0.1
## for calibration
r1<-0.1; eff1<-0.9

#Simulation::
X  <- simulateState(a, S0, F, Q, TT)
Yid  <- simulateObs(X, Z, Vi, mc, Vc, r=0)
Yre  <- simulateObs(X, Z, Vi, mc, Vc, ract)
\end{lstlisting}
\end{frame}
% ------------------------------------------------------------------------
\begin{frame}[fragile]%Frame 17
  \frametitle{Demonstration: {\tt rLSfilter} II}
\begin{lstlisting}
### calibration b
#limiting S_{t|t-1}
SS <- limitS(S, F, Q, Z, Vi)
# by efficiency in the ideal model
(B1 <- rLScalibrateB(eff=eff1, S=SS, Z=Z, V=Vi))
# by contamination radius
(B2 <- rLScalibrateB(r=r1, S=SS, Z=Z, V=Vi))

### evaluation of rLS
rerg1.id <- rLSFilter(Yid, a, Ss, F, Q, Z, Vi, B1$b)
rerg1.re <- rLSFilter(Yre, a, Ss, F, Q, Z, Vi, B1$b)
rerg2.id <- rLSFilter(Yid, a, Ss, F, Q, Z, Vi, B2$b)
rerg2.re <- rLSFilter(Yre, a, Ss, F, Q, Z, Vi, B2$b)
\end{lstlisting}
\end{frame}
\begin{frame}%Frame 18
\parbox{13cm}{\hspace{-0.5cm}\parbox{12cm}{\includegraphics[height=8cm, width=12cm]{rLS}}
\hspace{-11.2cm}\parbox{3.8cm}{%
\begin{tabular}{lp{3.5cm}}
\\[-3.5cm]
\multicolumn{2}{l}{\normalsize ideal situation}\\[3.1cm]
\small black:& \small real state,\\
\small red:& \small class. Kalman filter\\
\end{tabular}}%
\hspace{2.2cm}\parbox{3.8cm}{%
\begin{tabular}{lp{3.5cm}}
\\[-3.5cm]
\multicolumn{2}{l}{\normalsize AO-contaminated situation}\\[3.1cm]
\small green:& \small rLS filter ({\tt B1}),\\
\small blue:& \small rLS filter ({\tt B2})\normalsize
\end{tabular}}%
}
\end{frame}
% ------------------------------------------------------------------------
% Bibliography
% ------------------------------------------------------------------------
\begin{frame}%Frame 19
  \frametitle<presentation>{Bibliography}
  \begin{thebibliography}{10}
\begin{footnotesize}
  \beamertemplatebookbibitems
  % Start with overview books.
 \bibitem [Durbin/Koopman(01)]{D:K:01} Durbin, J. and Koopman, S. J.(2001):
 \newblock \emph{Time Series Analysis by
   State Space Methods.}
   \newblock Oxford University Press.
  \bibitem [P.R.(01)]{PR01} Ruckdeschel, P. (2001):
  \newblock \emph {Ans{\"a}tze zur Robustifizierung des Kalman Filters.}
  \newblock Bayreuther Mathematische Schriften, Vol. 64.
  \bibitem [R Development Core Team (2005)]{RMANUAL} {\sf R} Development Core Team (2006):
  \newblock \emph {{{\sf R:}{\em  A language and environment for statistical computing\/}.}}
\newblock  R Foundation for Statistical Computing, Vienna, Austria.\\
  {\href{http://www.R-project.org}{\tiny\url{http://www.R-project.org}}}
  \beamertemplatearticlebibitems
  \bibitem [Gilbert(05)]{Gil:05} Gilbert, P. (2005):
  \newblock Brief User's Guide: Dynamic Systems Estimation (DSE).
  \newblock  {\tiny Available in the file {\tt doc/dse-guide.pdf} distributed together with the {\sf R} bundle
  {\tt dse}, to be downloaded from \href{http://cran.r-project.org}{\url{http://cran.r-project.org}}}
  \bibitem [Martin(79)]{Mar:79} Martin, D. (1979):
  \newblock Approximate conditional-mean type smoothers and interpolators.
  \newblock In \emph {Smoothing techniques for curve estimation.\/}
  \newblock  Proc. Workshop Heidelberg 1979. Lect. Notes Math.~757, p.~117-143
\end{footnotesize}
  \end{thebibliography}
\end{frame}
% ------------------------------------------------------------------------
\end{document}
